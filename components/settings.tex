% !TEX root = ../main.tex
% Included by MAIN.TEX

%--------------------------------------------------
% Fonts and page setup
%--------------------------------------------------

% Default font
\usepackage{palatino}

% Enable special PostScript fonts (optional)
% \usepackage{pifont}

% Manipulate the footer
\usepackage{scrpage2}
\usepackage{scrhack}
\pagestyle{scrheadings}
\ifoot[\footertext]{\footertext} % \footertext set in INFO.TEX

% Set the font for the section headings
\renewcommand{\sectfont}{\normalfont \bfseries}

% Conditional commands in LaTeX documents, used for the \clearemptydoublepage.
\usepackage{ifthen}

% Typeset text in multiple columns (optional)
% \usepackage{multicol}

% Rotation tools, including rotated full-page floats (optional)
\usepackage{rotating}


%--------------------------------------------------
% Document structure
%--------------------------------------------------

% Create glossaries and lists of acronyms
\usepackage[toc, xindy]{glossaries}

% Standard LaTeX package for creating indexes
\usepackage{makeidx}


%--------------------------------------------------
% Bibliography
%--------------------------------------------------

% Set the bibliography style (default: plain)
\bibliographystyle{plain}

% Special biblography package (nice to have)
% \usepackage{natbib}


%--------------------------------------------------
% Graphics and floats
%--------------------------------------------------

% Enhanced support for graphics (recommended)
\usepackage{graphicx}
% Path to the figures directory (default: {figures/})
% Multiple entries are allowed, e.g. {{figures1/}{figures2/}}.
\graphicspath{{figures/}}

% Improved interface for floating objects (optional)
\usepackage{float}

% To use the subfigures (optional)
\usepackage{subcaption}


%--------------------------------------------------
% Mathematics
%--------------------------------------------------

% AMS mathematical facilities for LaTeX (recommended)
\usepackage{amsmath}

% TeX fonts from the American Mathematical Society (recommended)
\usepackage{amsfonts}

% Some extra math symbols (optional)
% \usepackage{amssymb}

% Extended maths fonts for LaTeX (optional)
% \usepackage{yhmath}

% Provide math delimiters whose size can be computed automatically (optional)
% \usepackage{commath}


%--------------------------------------------------
% Source code and algorithms
%--------------------------------------------------

% Source code typesetting
% \usepackage{listings} % (optional - alternative)
\usepackage[newfloat]{minted} % (recommended)
% Set global Minted options
\setminted{linenos, autogobble, frame=lines, framesep=2mm}
% Inline C++ (optional)
\newcommand{\incpp}[1]{\mintinline{c++}{#1}}
\newenvironment{code}{\captionsetup{type=listing}}{}
\SetupFloatingEnvironment{listing}{name=Source Code}

% Typeset algorithms - pseudocode (optional)
% \usepackage{algorithmicx}
% \usepackage{algpseudocode}
% Normal arrow comments
% \algrenewcommand{\algorithmiccomment}[1]{\hfill$\rightarrow$ #1}


%--------------------------------------------------
% Tables
%--------------------------------------------------

% Tables (optional)
\usepackage{tabu}

% Add color to LaTeX tables (optional)
% \usepackage{colortbl}

% Create tabular cells spanning multiple rows (optional)
% \usepackage{multirow}


%--------------------------------------------------
% PDF output
%--------------------------------------------------

% Pro­duce hy­per­text links in the doc­u­ment (recommended)
\usepackage[colorlinks=true,linkcolor=red,citecolor=red,%
anchorcolor=red,urlcolor=red,bookmarks=true,%
bookmarksopen=true,bookmarksopenlevel=0,plainpages=false,%
bookmarksnumbered=true,hyperindex=false,pdfstartview=true%
]{hyperref}

% PDF Metadata
\hypersetup{
  pdftitle={\title},%
  pdfauthor={\author}
}

% Make thumbnails (optional)
% \usepackage{thumbpdf}


%--------------------------------------------------
% Other settings
%--------------------------------------------------

% Define commands that appear not to eat spaces (optional)
\usepackage{xspace}

% Use colors (optional)
\usepackage{color}
