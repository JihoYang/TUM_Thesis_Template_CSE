% !TEX root = ../main.tex
% Included by MAIN.TEX

% manipulate footer
\usepackage{scrpage2}
\usepackage{scrhack}
\pagestyle{scrheadings}
\ifoot[\footertext]{\footertext} % \footertext set in INFO.TEX

% Set header Font
\renewcommand{\sectfont}{\normalfont \bfseries}        
  

\usepackage[english]{babel}

\usepackage{styles/shortoverview}

\usepackage{xspace}

\setkomafont{pagenumber}{\normalfont\rmfamily}

%% allow sophisticated control structures
\usepackage{ifthen}

% use Palatino as default font
\usepackage{palatino}

% enable special PostScript fonts
\usepackage{pifont}

%special biblography package (nice to have)
%\usepackage{natbib} 

% make thumbnails
% \usepackage{thumbpdf}

%to use the subfigures
\usepackage{subcaption}

%code import
%\usepackage{listings}
\usepackage[newfloat]{minted}

% Set global minted options
\setminted{linenos, autogobble, frame=lines, framesep=2mm}
% Inline C++
\newcommand{\incpp}[1]{\mintinline{c++}{#1}}
\newenvironment{code}{\captionsetup{type=listing}}{}
\SetupFloatingEnvironment{listing}{name=Source Code}

\usepackage{colortbl}

\usepackage{multicol}

\usepackage{commath}

%package for pseudocode
\usepackage{algorithmicx}
\usepackage{algpseudocode}
%normal arrow comments
\algrenewcommand{\algorithmiccomment}[1]{\hfill$\rightarrow$ #1}

%% enable TUM symbols on title page
\usepackage{styles/tumlogo}


\usepackage{multirow}

%% use colors
\usepackage{color}

%%table package
\usepackage{tabu}

%% make fancy math
\usepackage{amsmath}
\usepackage{amsfonts}
\usepackage{amssymb}
\usepackage{textcomp}
\usepackage{yhmath} % fr die adots
%% mark text as preliminary

%% create an index
\usepackage{makeidx}

% for the program environment
\usepackage{float}

%for glossary
\usepackage[toc, xindy]{glossaries}

%----------------------------------------------------
%      Graphics and Hyperlinks
%----------------------------------------------------

%% check for pdfTeX
\ifx\pdftexversion\undefined

 %% use PostScript graphics
 \usepackage[dvips]{graphicx}

 \DeclareGraphicsExtensions{.eps,.epsi}
 \graphicspath{{figures/}{figures/review}}
 %% allow rotations
 \usepackage{rotating}
 %% mark pages as draft copies
 %\usepackage[english,all,light]{draftcopy}
 %% use hypertex version of hyperref
 \usepackage[hypertex,hyperindex=false,colorlinks=false]{hyperref}
\else %% reduce output size \pdfcompresslevel=9

 %% declare pdfinfo
 %\pdfinfo {
 %  /Title (my title)
 %  /Creator (pdfLaTeX)
 %  /Author (my name)
 %  /Subject (my subject	)
 %  /Keywords (my keywords)
 %}
 %% use pdf or jpg graphics

 \usepackage[pdftex]{graphicx}
 \DeclareGraphicsExtensions{.jpg,.JPG,.png,.pdf,.eps}
 \graphicspath{{figures/}}

 %% Load float package, for enabling floating extensions
 \usepackage{float}

 %% allow rotations
 \usepackage{rotating}
 %% use pdftex version of hyperref
 \usepackage[pdftex,colorlinks=true,linkcolor=red,citecolor=red,%
 anchorcolor=red,urlcolor=red,bookmarks=true,%
 bookmarksopen=true,bookmarksopenlevel=0,plainpages=false,%
 bookmarksnumbered=true,hyperindex=false,pdfstartview=true%
 ]{hyperref}
\fi

\bibliographystyle{plain}
